\chapter{Structure of $\Breal$ and $\Breal^{n}$}
\pagebreak[4]

\mysection{30}{Algebraic structure of $\Breal$ }
\subsection*{Remark on the Archimedean principle}
The Archimedean principle is sometimes stated as :\\
If $x,\,y\in \Breal$ with $y>0$, then $\exists n\in \Zed: (n-1)y\leq x < ny$.\\\\
On page $25$, we have \\\\
$\mathbf{30.6}$ \textbf{The least upper bound axiom for the real number system} $\mathbf{\Breal}$.\\
If $S$ is a nonempty subset of $\Breal$ and $S$ has an upper bound, then $S$ has a least upper bound in $\Breal$.\\\\
It is strange that  the author first gives the Archimedean principle as a principle (aka Axiom ?) and afterwards $\mathbf{30.6}$ as an axiom as the Archimedean principle can be proved with this axiom. \\\\
\textbf{Proof} (by contradiction):\\
Be arbitrary $x,\, y\in \Breal$ with $y> 0$. Suppose that we have $py\leq x$ is true for all $p\in \Zed$. Be $A$ the subset of $\Breal$ with all the real numbers $py$ as elements. From axiom $\mathbf{30.6}$ we know that $A$ has a least upper bound with $x$ as an upper bound. Be $\xi= lub(A)$. As we have $y>0$, then $\xi -y$ is not an upper bound of $A$.  This mean that there is a $p$ such that $py>\xi -y$. Hence, $ (p+1)y>\xi$. Thus $\xi$ can not be a upper bound of $A$, and we have a contradiction. Hence, there must be a $p\in \Zed$ for which $py>x$.
\newpage
\renewcommand{\thesubsection}{\thesection.\arabic{subsection}}
\setcounter{subsection}{0}
\subsection{}
%\subsubsection{}{}
\begin{tcolorbox}
Explain by example why it is that the system of all rational numbers does not satisfy the least upper bound axiom.
\end{tcolorbox}
Be $A\subset\Qiuu$ such that $A=\{x^2\leq 2,\, x\in \Qiuu\}$. $A$ has upper bounds (e.g.  $sup(A)=2$). This set can be re-expressed as $A= (-\sqrt{2},\sqrt{2})$ and has no $l.u.b.(A)$ as $\sqrt{2}\not \in \Qiuu$.
$$\blacklozenge$$

\subsection{}
%\subsubsection{}{}
\begin{tcolorbox}
Prove that the least upper bound property implies the following: If $S$ is a nonempty subset of real numbers that has a lower bound, then $S$ has a greatest lower bound. 
\end{tcolorbox}
Suppose $S$ is a nonempty subset of real numbers that has a lower bound $\mathbcal{s}$. For all $x\in S$ we have $\mathbcal{s}\leq x$. Let's define a new set $\hat{S}$ with $\hat{S}=\{2\mathbcal{s}-x: x\in S\}$. Then, $\mathbcal{s}$ is an upper bound for $\hat{S}$. By the upper bound axiom for real number, we have a $\mathbcal{\hat{s}}= l.u.b.(\hat{S})$ and we have $2\mathbcal{s}-x\leq \mathbcal{\hat{s}}$ for all $x\in S$. And thus, $$2\mathbcal{s}- \mathbcal{\hat{s}}\leq x$$
for all $x\in S$.\\
$2\mathbcal{s}- \mathbcal{\hat{s}}$ is a $g.l.b.(A)$ as, suppose we would have a $\epsilon>0$ such that  $2\mathbcal{s}- \mathbcal{\hat{s}}+\epsilon \leq x$ and thus $\underbrace{2\mathbcal{s}-x}_{\in\hat{S}} \leq \mathbcal{\hat{s}}-\epsilon$ for all $\hat{x}\in \hat{S}$. We get a contradiction as we would have $\mathbcal{\hat{s}}-\epsilon$ as an upper bound which is smaller than $l.u.b.(\hat{S})$.
$$\blacklozenge$$

\subsection{}
%\subsubsection{}{}
\begin{tcolorbox}
Let $S = \left\{x:x=1-\frac{1}{n},\, n\in\Pee\right\}$. Find $l.u.b. (S)$ and $g.l.b. (S)$ if they exist. 
\end{tcolorbox}
$$ $$ 
For $n=1$ we have $x=0$ and thus $g.l.b(S)=0$.\\\\
For $n\rightarrow +\infty $ we have $x\rightarrow 1$ and thus $l.u.b(S)=1$.
$$\blacklozenge$$

\subsection{}
%\subsubsection{}{}
\begin{tcolorbox}
Let $f:\Breal\rightarrow \Breal$  be given by $f (x)=x^3$. Find the $l.u.b. (f [\{x: 0 < x < 1\}])$. \\
Find $l.u.b. (f [\{x:0\leq x\leq 1\})$. 
\end{tcolorbox}
$$ $$ 
$l.u.b. (f [\{x: 0 < x < 1\}])= 1$.\\\\
$l.u.b. (f [\{x:0\leq x\leq 1\})= 1$
$$\blacklozenge$$

\subsection{}
%\subsubsection{}{}
\begin{tcolorbox}
Suppose that $f: \{x :0 < x\}\rightarrow\Breal$ is given by $f (x)= \frac{1}{x}$  for $0 < x$. Does $f [\{x : 0 < x\}]$ have a $l.u.b.$? Does it have a $g.l.b.$? 
\end{tcolorbox}
$$ $$ 
$l.u.b. (f [\{x : 0 < x\}])$ does not exist.\\\\
$g.l.b. (f [\{x : 0 < x\}])=0$.
$$\blacklozenge$$

\subsection{}
%\subsubsection{}{}
\begin{tcolorbox}
Give an example of a function $f$ defined on a closed interval $S$ such that $l.u.b. (f [S])$ exists but $f$ does not attain a maximum value on $S$.
\end{tcolorbox}
Consider $f:\Breal_{[0,1]}\rightarrow \Breal$ such that $f=\left\{\frac{1-a^{-\frac{x}{1-x}}}{1+a^{-\frac{x}{1-x}}}:x\in \Breal_{[0,1]},\, a>1 \right\}$ with $l.u.b. (f [S])=1\not\in f [S]$.
$$\blacklozenge$$

\subsection{}
%\subsubsection{}{}
\begin{tcolorbox}
Prove the following statement: If $a = l.u.b. (A)$, then for each $\epsilon > 0$ , there is an $x \in  A$ such that $a-\epsilon < x \leq a$. State and prove an analogous proposition for $g.l.b. (A)$. 
\end{tcolorbox}
Suppose for a given $\epsilon$,  $\not\exists\, x\in A:\, a-\epsilon < x$, then, $x\leq a-\epsilon$. Thus $a-\epsilon$ is a lower bound which is smaller than $l.u.b.(S)$. We have a contradiction and thus $\forall\epsilon\in\Breal, \exists\, x\in A:\, a-\epsilon < x \leq a$.\\\\
For $b= g.l.b. (A)$, the statement becomes $$\forall\epsilon\in\Breal, \exists\, x\in A:\, b\leq  x < b+\epsilon$$
Suppose for a given $\epsilon$,  $\not\exists\, x\in A:\, x < b+\epsilon$, then, $x \leq b+\epsilon$. Thus $b+\epsilon$ is a upper bound which is greater than $g.l.b.(S)$. We have a contradiction and thus $\forall\epsilon\in\Breal, \exists\, x\in A:\, b\leq  x < b+\epsilon$.
$$\blacklozenge$$

\subsection{}
%\subsubsection{}{}
\begin{tcolorbox}

Prove that if $A$ is a nonempty bounded set of real numbers then $g.l.b. (A) \leq l.u.b. (A)$. For what kind of set $A$ is $g.l.b. (A) =l.u.b. (A)$? 
\end{tcolorbox}
$A$ is bounded, hence by the upper bound property and its corollary (see Exercise 2.30.2) then $A$ has both a $l.u.b.(A)$  and a $g.l.b.(A)$. This mean that for every $x\in A$ we have $l.u.b.(A)\leq x\leq g.l.b.(A)$ and thus $l.u.b.(A)\leq g.l.b.(A)$\\\\
We can have $g.l.b. (A) =l.u.b. (A)$ when $A$ is a singleton, i.e. $A=\{x\}$. E.g. $A=f[\Breal]$ with $f=\{c\}$ with $c$ a constant, has $g.l.b. (A) =l.u.b. (A)=c$ .
$$\blacklozenge$$



\subsection{}
%\subsubsection{}{}
\begin{tcolorbox}
Prove that if $A$ and $B$ are nonempty subsets of $\Breal$ and $A \subset B$, then $g.l.b. (B) \leq g.l.b. (A)\leq  l.u.b. (A) \leq l.u.b. (B)$. 
\end{tcolorbox}
From $\mathbf{2.30.8}$ we know that $l.u.b.(A)\leq g.l.b.(A)$ and $l.u.b.(B)\leq g.l.b.(B)$. \\\\
Suppose we have $g.l.b.(A)< g.l.b.(B)$. This implies that $\exists x\in A: x\not \in B$. We have a contradiction as $A\subset B$ and thus we must have $g.l.b.(B)\leq g.l.b.(A)$.\\
In the same vein, suppose we have $l.u.b.(B)< l.u.b.(A)$. This implies that $\exists x\in A: x\not \in B$. We have a contradiction as $A\subset B$ and thus we must have $l.u.b.(A)\leq l.u.b.(B)$. \\
Putting this all together we get indeed
$$g.l.b. (B) \leq g.l.b. (A)\leq  l.u.b. (A) \leq l.u.b. (B)$$
$$\blacklozenge$$
\newpage

\mysection{31}{ Distance between two points in $\Breal$}

\renewcommand{\thesubsection}{\thesection.\arabic{subsection}}
\setcounter{subsection}{0}
\subsection{}
%\subsubsection{}{}
\begin{tcolorbox}
Verify the properties stated in $\mathbf{31.2}$
\end{tcolorbox}
$\mathbf{31.2(a)}$ and $\mathbf{31.2(b)}$ are trivial as $d(x,y)=|x-y|$.\\\\
$\mathbf{31.2(c)}$:\\
We consider 4 possibilities (the cases $a=b$ or $b=c$ are excluded as in that case $d(a,b)$ or $d(b,c)$ are zero).\\
i) $a<b$ and $b<c$. Then, 
$$d(a,c)= |a-b+b-c| = |-|a-b|-|b-c||= | |a-b|+|b-c| |= d(a,b)+d(b,c)$$
ii) $a<b$ and $b>c$. Then, 
$$d(a,c)= |a-b+b-c| = |-|a-b|+|b-c||< | |a-b|+|b-c| |= d(a,b)+d(b,c)$$
ii) $a>b$ and $b<c$. Then, 
$$d(a,c)= |a-b+b-c| = ||a-b|-|b-c||< | |a-b|+|b-c| |= d(a,b)+d(b,c)$$
iv) $a>b$ and $b>c$. Then, 
$$d(a,c)= |a-b+b-c| = ||a-b|+|b-c||= | |a-b|+|b-c| |= d(a,b)+d(b,c)$$
and conclude $$d(a,b)+d(b,c) \geq d(a,c)$$
$$\blacklozenge$$

\subsection{}
%\subsubsection{}{}
\begin{tcolorbox}
Let $p\in\Breal$. Give an example of a collection $\mathscr{K}$ of neighborhoods of $p$ such that $\bigcap\mathscr{K}$ is not a neighborhood. Show that if $\mathscr{K}$ is a nonempty finite collection of neighborhoods of $p$, then $\bigcup\mathscr{K}$ and $\bigcap\mathscr{K}$ are neighborhoods of $p$.
\end{tcolorbox}
Consider the infinite collection $\mathscr{K}=\{N(p;\frac{1}{n}):n\in \Pee\}$, then $\bigcap\mathscr{K}=\{p\}$ and no $\epsilon>0$ can be found for which $N(p;\epsilon)$ exists. \\\\
If $\mathscr{K}$ is a finite collection then $\mathscr{K}$ is countable with element $N(p;\epsilon_i)$ and we can construct a finite set $$\mathscr{E}=\{\epsilon_i:i=1,2,\dots,n\}$$  As $ \mathscr{E}\subset \Breal$ and by the least upper bound principle, $\mathscr{E}$ has both a $\epsilon_u= l.u.b.(\mathscr{E})$ and a $\epsilon_l = g.l.b(\mathscr{E})$. Moreover, $N(p;\epsilon_u)$ and  $N(p;\epsilon_l)$ are elements of $\mathscr{K}$ and hence we get\\\\
$$\bigcup\mathscr{K}=N(p;\epsilon_u)$$ and $$\bigcap\mathscr{K}=N(p;\epsilon_l)$$.
$$\blacklozenge$$
\newpage


\mysection{32}{Limit of a sequence in $\Breal$}

\renewcommand{\thesubsection}{\thesection.\arabic{subsection}}
\setcounter{subsection}{0}
\subsection{}
%\subsubsection{}{}
\begin{tcolorbox}
Suppose that $(a_n)$ is a sequence such that $a_n\leq a_{n+1}$, $(a_{n+1}\leq a_{n})$ for each positive integer $n$. Suppose further that the sequence $(a_n)$ is bounded above (below). Then $\lim (a_n)$ exists. 
\end{tcolorbox}
$$ $$ 
$\mathbf{a_n\leq a_{n+1}}$\\\\
As $(a_n)$ is bounded above and is a subset of $\Breal$, then by the upper bound property of $\Breal$, has a lower upper bound, $A=l.u.b.$.\\
We first notice that for any arbitrary $\epsilon\in \Breal$, there must be a $a_N$ such that $A-\epsilon < a_N\leq A$ (as otherwise $A-\epsilon$ would be an upper bound, giving a contradiction).\\
So we have $A-\epsilon < a_N$ and thus $A-a_N < \epsilon$.  Furthermore as $a_{N+k}\geq a_N$ for all $k$ we will have $A-a_{N+k}\leq A-a_N<\epsilon$.  As $\forall n, A\geq a_n $, $A-a_n$ will stay positive for all $n$. Thus, the last result can be written as $|A-a_{N+k}|<\epsilon$ for a chosen $\epsilon$. From the definition of the limit, we conclude that the $\lim (a_n)= l.u.b.(a_n)$ exists.\\\\
$\mathbf{a_{n+1}\leq a_{n}}$\\\\
As $(a_n)$ is bounded below  and is a subset of $\Breal$, then by the corollary of the upper bound property of $\Breal$, has a greatest lower upper bound, $L=g.l.b.$.\\
We first notice that for any arbitrary $\epsilon\in \Breal$, there must be a $a_N$ such that $L\leq a_N < L+\epsilon$ (as otherwise $L+\epsilon$ would be a lower bound, giving a contradiction).\\
So we have $a_N < L+\epsilon$ and thus $a_N -L < \epsilon$.  Furthermore as $a_{N+k}\leq a_N$ for all $k$ we will have $a_{N+k}-L\leq a_{N}-L<\epsilon$.  As $\forall n, L\leq a_n $, $a_n-L$ will stay positive for all $n$. Thus, the last result can be written as $|L-a_{N+k}|<\epsilon$ for a chosen $\epsilon$. From the definition of the limit, we conclude that the $\lim (a_n)= g.l.b.(a_n)$ exists.
$$\blacklozenge$$

\subsection{}
%\subsubsection{}{}
\begin{tcolorbox}
Let the sequence $(c_n)$ in $\Breal$ be given by $c_n = a_n+b_n$, where $\lim(a_n)= A$ and $\lim(b_n)= B$.\\
 Then, $\lim (c_n)= A +B$.
\end{tcolorbox}
From the definition of the limit of a sequence, for a given $\frac{\epsilon}{2}$ there will be elements $a_N$ and $b_{N^{'}}$ such that for all $n\geq N$ and $n^{'}\geq {N^{'}}$ we will have 
$$|A-a_n|<\frac{\epsilon}{2}\Et |B-b_{n^{'}}|<\frac{\epsilon}{2}$$
Let's put $M=sup(N,\,N^{'})$. 
Adding the two inequalities gives 
$$|A-a_m|+|B-b_{m^{}}|<\epsilon$$
for any $m\geq M$.
The triangle inequality of the distance in $\Breal$ states
$$d(x,z)\leq d(x,y)+d(y,z)$$
Put $x= a_m-B,\, z= A-b_m$ and $y= A-B$. The triangle inequality gives $|a_m-B-A+b_m  | \leq |a_m-B-A+B |+|A-b_m -A+B |$ or
$$|a_m+b_m-(A+B)| \leq |A-a_m |+|B-b_m |$$
But $|A-a_m|+|B-b_{m^{}}|<\epsilon$ and get 
$$|a_m+b_m-(A+B)| <\epsilon$$
giving $\lim (a_n+b_n)= A+B$
$$\blacklozenge$$

\subsection{}
%\subsubsection{}{}
\begin{tcolorbox}
 Let the sequence $(c_n)$ in $\Breal$ be given by $c_n = ka_n$, where $k$ is a constant and $\lim (a_n)= A$.\\
 Then, $\lim (c_n)= kA$.
\end{tcolorbox}
Be given $k$. Let's chose a $\frac{\epsilon}{|k|}$ (we suppose $k\neq 0$, this case being trivial).\\
We have $|a_n-A|<\frac{\epsilon}{|k|}$ for all $n\geq$ then  certain $N$. Multiplying by $|k|$ gives $|k||a_n-A|<\frac{|k|\epsilon}{|k|}$ and thus
$$|ka_n-kA|<\epsilon$$
and conclude $\lim ka_n= kA$.
$$\blacklozenge$$

\subsection{}
%\subsubsection{}{}
\begin{tcolorbox}
 Let the sequence $(c_n)$ in $\Breal$ be given by $c_n = a_nb_n$, where $\lim (a_n)= A$ and $\lim (b_n)= B$.\\
 Then, $\lim (c_n)= AB$. 
\end{tcolorbox}

$$\blacklozenge$$

\subsection{}
%\subsubsection{}{}
\begin{tcolorbox}
Let the sequence $(c_n)$ in $\Breal$ be given by $c_n = \frac{a_n}{b_n}$, where $b_n\neq 0$,  $\lim (a_n)= A$ and $\lim (b_n)= B\neq 0$.\\
 Then, $\lim (c_n)= \frac{A}{B}$.
\end{tcolorbox}

$$\blacklozenge$$

\subsection{}
%\subsubsection{}{}
\begin{tcolorbox}
If a sequence $(a_i)$ has a limit, it is unique.
\end{tcolorbox}

$$\blacklozenge$$
\newpage