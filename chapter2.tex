\chapter{Structure of $\Breal$ and $\Breal^{n}$}
\pagebreak[4]

\mysection{30}{Algebraic structure of $\Breal$ }
\subsection*{Remark on the Archimedean principle}
The Archimedean principle is sometimes stated as :\\
If $x,\,y\in \Breal$ with $y>0$, then $\exists n\in \Zed: (n-1)y\leq x < ny$.\\\\
On page $25$, we have \\\\
$\mathbf{30.6}$ \textbf{The least upper bound axiom for the real number system} $\mathbf{\Breal}$.\\
If $S$ is a nonempty subset of $\Breal$ and $S$ has an upper bound, then $S$ has a least upper bound in $\Breal$.\\\\
It is strange that  the author first gives the Archimedean principle as a principle (aka Axiom ?) and afterwards $\mathbf{30.6}$ as an axiom as the Archimedean principle can be proved with this axiom. \\\\
\textbf{Proof} (by contradiction):\\
Be arbitrary $x,\, y\in \Breal$ with $y> 0$. Suppose that we have $py\leq x$ is true for all $p\in \Zed$. Be $A$ the subset of $\Breal$ with all the real numbers $py$ as elements. From axiom $\mathbf{30.6}$ we know that $A$ has a least upper bound with $x$ as an upper bound. Be $\xi= lub(A)$. As we have $y>0$, then $\xi -y$ is not an upper bound of $A$.  This mean that there is a $p$ such that $py>\xi -y$. Hence, $ (p+1)y>\xi$. Thus $\xi$ can not be a upper bound of $A$, and we have a contradiction. Hence, there must be a $p\in \Zed$ for which $py>x$.
\newpage
\renewcommand{\thesubsection}{\thesection.\arabic{subsection}}
\setcounter{subsection}{0}
\subsection{}
%\subsubsection{}{}
\begin{tcolorbox}
Explain by example why it is that the system of all rational numbers does not satisfy the least upper bound axiom.
\end{tcolorbox}
Be $A\subset\Qiuu$ such that $A=\{\frac{2^n-1}{2^n}: n\in \Pee\}$. $A$ has an upper bound $sup(A)=1$ as $\frac{2^n-1}{2^n}<1,\,\forall n\in \Pee$. Suppose we have a lower upper bound, then we must have a $N\in \Pee$ such that $l.u.b.(A)= \frac{2^N-1}{2^N}$. But then, we have $\frac{2^{N+1}-1}{2^{N+1}}= \frac{2^N-1}{2^N}-\frac{1}{2^{N-1}}$. \\
$\frac{2^N-1}{2^N}-\frac{1}{2^{N-1}}$ is still a rational number but $\frac{2^N-1}{2^N}-\frac{1}{2^{N-1}}= l.u.b.(A)-\frac{1}{2^{N-1}}< l.u.b.(A)$. Hence, we can't find a $N\in\Pee$ such that $\frac{2^N-1}{2^N}=l.u.b.(A)$ and $\Qiuu$ does not satisfy the upper bound axiom.\\\\
\textbf{Remark}: A very superficial conclusion could be that, because $A\subset \Breal$ as well, this counter example would mean that the upper bound axiom is not satisfied for $\Breal$. The pint here is that to find the $l.u.b.(A)$ we are not restricted to numbers of the form $\frac{2^n-1}{2^n}: n\in \Pee$  
$$\blacklozenge$$

\subsection{}
%\subsubsection{}{}
\begin{tcolorbox}
Prove that the least upper bound property implies the following: If $S$ is a nonempty subset of real numbers that has a lower bound, then $S$ has a greatest lower bound. 
\end{tcolorbox}
Suppose $S$ is a nonempty subset of real numbers that has a lower bound $\mathbcal{s}$. For all $x\in S$ we have $\mathbcal{s}\leq x$. Let's define a new set $\hat{S}$ with $\hat{S}=\{2\mathbcal{s}-x: x\in S\}$. Then, $\mathbcal{s}$ is an upper bound for $\hat{S}$. By the upper bound axiom for real number, we have a $\mathbcal{\hat{s}}= l.u.b.(\hat{S})$ and we have $2\mathbcal{s}-x\leq \mathbcal{\hat{s}}$ for all $x\in S$. And thus, $$2\mathbcal{s}- \mathbcal{\hat{s}}\leq x$$
for all $x\in S$.\\
$2\mathbcal{s}- \mathbcal{\hat{s}}$ is a $g.l.b.(A)$ as, suppose we would have a $\epsilon>0$ such that  $2\mathbcal{s}- \mathbcal{\hat{s}}+\epsilon \leq x$ and thus $\underbrace{2\mathbcal{s}-x}_{\in\hat{S}} \leq \mathbcal{\hat{s}}-\epsilon$ for all $\hat{x}\in \hat{S}$. We get a contradiction as we would have $\mathbcal{\hat{s}}-\epsilon$ as an upper bound which is smaller than $l.u.b.(\hat{S})$.
$$\blacklozenge$$

\subsection{}
%\subsubsection{}{}
\begin{tcolorbox}
Let $S = \left\{x:x=1-\frac{1}{n},\, n\in\Pee\right\}$. Find $l.u.b. (S)$ and $g.l.b. (S)$ if they exist. 
\end{tcolorbox}

$$\blacklozenge$$

\subsection{}
%\subsubsection{}{}
\begin{tcolorbox}
Let $f:\Breal\rightarrow \Breal$  be given by $f (x)=x^3$. Find the $1.u.b. (f [\{x: 0 < x < 1\}])$. \\
Find $l.u.b. (f [\{x:0\leq x\leq 1\})$. 
\end{tcolorbox}

$$\blacklozenge$$

\subsection{}
%\subsubsection{}{}
\begin{tcolorbox}
Suppose that $f: \{x :0 < x\}\rightarrow\Breal$ is given by $f (x)= \frac{1}{x}$  for $0 < x$. Does $f [\{x : 0 < x\}]$ have a $l.u.b.$? Does it have a $g.l.b.$? 
\end{tcolorbox}

$$\blacklozenge$$

\subsection{}
%\subsubsection{}{}
\begin{tcolorbox}
Give an example of a function $f$ defined on a closed interval $S$ such that $l.u.b. (f [S])$ exists but $f$ does not attain a maximum value on $S$.
\end{tcolorbox}

$$\blacklozenge$$

\subsection{}
%\subsubsection{}{}
\begin{tcolorbox}
Prove the following statement: If$a = I.u.b. (A)$, then for each $\epsilon > 0$ , there is an $x \in  A$ such that $a-\epsilon < x \leq a$. State and prove an analogous proposition for $g.l.b. (A)$. 
\end{tcolorbox}

$$\blacklozenge$$

\subsection{}
%\subsubsection{}{}
\begin{tcolorbox}

Prove that if $A$ is a nonempty bounded set of real numbers then $g.l.b. (A) \leq l.u.b. (A)$. For what kind of set $A$ is $g.l.b. (A) =1.u.b. (A)$? 
\end{tcolorbox}

$$\blacklozenge$$



\subsection{}
%\subsubsection{}{}
\begin{tcolorbox}
Prove that if $A$ and $B$ are nonempty subsets of $\Breal$ and $A \subset B$, then $g.l.b. (B) \leq g.l.b. (A)\leq  l.u.b. (A) \leq l.u.b. (B)$.
\end{tcolorbox}

$$\blacklozenge$$
\newpage

\mysection{31}{ Distance between two points in $\Breal$}

\renewcommand{\thesubsection}{\thesection.\arabic{subsection}}
\setcounter{subsection}{0}
\subsection{}
%\subsubsection{}{}
\begin{tcolorbox}
Verify the properties stated in $\mathbf{31.2}$
\end{tcolorbox}

$$\blacklozenge$$

\subsection{}
%\subsubsection{}{}
\begin{tcolorbox}
Let $p\in\Breal$. Give an example of a collection $\mathscr{K}$ of neighborhoods of $p$ such that $\bigcap\mathscr{K}$ is not a neighborhood. Show that if $\mathscr{K}$ is a nonempty finite collection of neighborhoods of $p$, then $\bigcup\mathscr{K}$ and $\bigcap\mathscr{K}$ are neighborhoods of $p$.
\end{tcolorbox}

$$\blacklozenge$$
\newpage


\mysection{32}{Limit of a sequence in $\Breal$}

\renewcommand{\thesubsection}{\thesection.\arabic{subsection}}
\setcounter{subsection}{0}
\subsection{}
%\subsubsection{}{}
\begin{tcolorbox}
Suppose that $(a_n)$ is a sequence such that $a_n\leq a_{n+1}$, $a_{n+1}\leq a_{n}$ for each positive integer $n$. Suppose further that the sequence $(a_n)$ is bounded above (below). Then $\lim (a_n)$ exists. 
\end{tcolorbox}

$$\blacklozenge$$

\subsection{}
%\subsubsection{}{}
\begin{tcolorbox}
Let the sequence $(c_n)$ in $\Breal$ be given by $c_n = a_n+b_n$, where $\lim(a_n)= A$ and $\lim(b_n)= B$.\\
 Then, $\lim (c_n)= A +B$.
\end{tcolorbox}

$$\blacklozenge$$

\subsection{}
%\subsubsection{}{}
\begin{tcolorbox}
 Let the sequence $(c_n)$ in $\Breal$ be given by $c_n = ka_n$, where $k$ is a constant and $\lim (a_n)= A$.\\
 Then, $\lim (c_n)= kA$.
\end{tcolorbox}

$$\blacklozenge$$

\subsection{}
%\subsubsection{}{}
\begin{tcolorbox}
 Let the sequence $(c_n)$ in $\Breal$ be given by $c_n = a_nb_n$, where $\lim (a_n)= A$ and $\lim (b_n)= B$.\\
 Then, $\lim (c_n)= AB$. 
\end{tcolorbox}

$$\blacklozenge$$

\subsection{}
%\subsubsection{}{}
\begin{tcolorbox}
Let the sequence $(c_n)$ in $\Breal$ be given by $c_n = \frac{a_n}{b_n}$, where $b_n\neq 0$,  $\lim (a_n)= A$ and $\lim (b_n)= B\neq 0$.\\
 Then, $\lim (c_n)= \frac{A}{B}$.
\end{tcolorbox}

$$\blacklozenge$$

\subsection{}
%\subsubsection{}{}
\begin{tcolorbox}
If a sequence $(a_i)$ has a limit, it is unique.
\end{tcolorbox}

$$\blacklozenge$$
\newpage