\chapter{Structure of $\Breal$ and $\Breal^{n}$}
\pagebreak[4]

\mysection{30}{Algebraic structure of $\Breal$ }
\subsection*{Remark on the Archimedean principle}
The Archimedean principle is sometimes stated as :\\
If $x,\,y\in \Breal$ with $y>0$, then $\exists n\in \Zed: (n-1)y\leq x < ny$.\\\\
On page $25$, we have \\\\
$\mathbf{30.6}$ \textbf{The least upper bound axiom for the real number system} $\mathbf{\Breal}$.\\
If $S$ is a nonempty subset of $\Breal$ and $S$ has an upper bound, then $S$ has a least upper bound in $\Breal$.\\\\
It is strange that  the author first gives the Archimedean principle as a principle (aka Axiom ?) and afterwards $\mathbf{30.6}$ as an axiom as the Archimedean principle can be proved with this axiom. \\\\
\textbf{Proof} (by contradiction):\\
Be arbitrary $x,\, y\in \Breal$ with $y> 0$. Suppose that we have $py\leq x$ is true for all $p\in \Zed$. Be $A$ the subset of $\Breal$ with all the real numbers $py$ as elements. From axiom $\mathbf{30.6}$ we know that $A$ has a least upper bound with $x$ as an upper bound. Be $\xi= lub(A)$. As we have $y>0$, then $\xi -y$ is not an upper bound of $A$.  This mean that there is a $p$ such that $py>\xi -y$. Hence, $ (p+1)y>\xi$. Thus $\xi$ can not be a upper bound of $A$, and we have a contradiction. Hence, there must be a $p\in \Zed$ for which $py>x$.
\newpage
\renewcommand{\thesubsection}{\thesection.\arabic{subsection}}
\setcounter{subsection}{0}
\subsection{}
%\subsubsection{}{}
\begin{tcolorbox}
Explain by example why it is that the system of all rational numbers does not satisfy the least upper bound axiom.
\end{tcolorbox}
Be $A\subset\Qiuu$ such that $A=\{x^2\leq 2,\, x\in \Qiuu\}$. $A$ has upper bounds (e.g.  $sup(A)=2$). This set can be re-expressed as $A= (-\sqrt{2},\sqrt{2})$ and has no $l.u.b.(A)$ as $\sqrt{2}\not \in \Qiuu$.
$$\blacklozenge$$

\subsection{}
%\subsubsection{}{}
\begin{tcolorbox}
Prove that the least upper bound property implies the following: If $S$ is a nonempty subset of real numbers that has a lower bound, then $S$ has a greatest lower bound. 
\end{tcolorbox}
Suppose $S$ is a nonempty subset of real numbers that has a lower bound $\mathbcal{s}$. For all $x\in S$ we have $\mathbcal{s}\leq x$. Let's define a new set $\hat{S}$ with $\hat{S}=\{2\mathbcal{s}-x: x\in S\}$. Then, $\mathbcal{s}$ is an upper bound for $\hat{S}$. By the upper bound axiom for real number, we have a $\mathbcal{\hat{s}}= l.u.b.(\hat{S})$ and we have $2\mathbcal{s}-x\leq \mathbcal{\hat{s}}$ for all $x\in S$. And thus, $$2\mathbcal{s}- \mathbcal{\hat{s}}\leq x$$
for all $x\in S$.\\
$2\mathbcal{s}- \mathbcal{\hat{s}}$ is a $g.l.b.(A)$ as, suppose we would have a $\epsilon>0$ such that  $2\mathbcal{s}- \mathbcal{\hat{s}}+\epsilon \leq x$ and thus $\underbrace{2\mathbcal{s}-x}_{\in\hat{S}} \leq \mathbcal{\hat{s}}-\epsilon$ for all $\hat{x}\in \hat{S}$. We get a contradiction as we would have $\mathbcal{\hat{s}}-\epsilon$ as an upper bound which is smaller than $l.u.b.(\hat{S})$.
$$\blacklozenge$$

\subsection{}
%\subsubsection{}{}
\begin{tcolorbox}
Let $S = \left\{x:x=1-\frac{1}{n},\, n\in\Pee\right\}$. Find $l.u.b. (S)$ and $g.l.b. (S)$ if they exist. 
\end{tcolorbox}
$$ $$ 
For $n=1$ we have $x=0$ and thus $g.l.b(S)=0$.\\\\
For $n\rightarrow +\infty $ we have $x\rightarrow 1$ and thus $l.u.b(S)=1$.
$$\blacklozenge$$

\subsection{}
%\subsubsection{}{}
\begin{tcolorbox}
Let $f:\Breal\rightarrow \Breal$  be given by $f (x)=x^3$. Find the $l.u.b. (f [\{x: 0 < x < 1\}])$. \\
Find $l.u.b. (f [\{x:0\leq x\leq 1\})$. 
\end{tcolorbox}
$$ $$ 
$l.u.b. (f [\{x: 0 < x < 1\}])= 1$.\\\\
$l.u.b. (f [\{x:0\leq x\leq 1\})= 1$
$$\blacklozenge$$

\subsection{}
%\subsubsection{}{}
\begin{tcolorbox}
Suppose that $f: \{x :0 < x\}\rightarrow\Breal$ is given by $f (x)= \frac{1}{x}$  for $0 < x$. Does $f [\{x : 0 < x\}]$ have a $l.u.b.$? Does it have a $g.l.b.$? 
\end{tcolorbox}
$$ $$ 
$l.u.b. (f [\{x : 0 < x\}])$ does not exist.\\\\
$g.l.b. (f [\{x : 0 < x\}])=0$.
$$\blacklozenge$$

\subsection{}
%\subsubsection{}{}
\begin{tcolorbox}
Give an example of a function $f$ defined on a closed interval $S$ such that $l.u.b. (f [S])$ exists but $f$ does not attain a maximum value on $S$.
\end{tcolorbox}
Consider $f:\Breal_{[0,1]}\rightarrow \Breal$ such that $f=\left\{\frac{1-a^{-\frac{x}{1-x}}}{1+a^{-\frac{x}{1-x}}}:x\in \Breal_{[0,1]},\, a>1 \right\}$ with $l.u.b. (f [S])=1\not\in f [S]$.
$$\blacklozenge$$

\subsection{}
%\subsubsection{}{}
\begin{tcolorbox}
Prove the following statement: If $a = l.u.b. (A)$, then for each $\epsilon > 0$ , there is an $x \in  A$ such that $a-\epsilon < x \leq a$. State and prove an analogous proposition for $g.l.b. (A)$. 
\end{tcolorbox}
Suppose for a given $\epsilon$,  $\not\exists\, x\in A:\, a-\epsilon < x$, then, $x\leq a-\epsilon$. Thus $a-\epsilon$ is a lower bound which is smaller than $l.u.b.(S)$. We have a contradiction and thus $\forall\epsilon\in\Breal, \exists\, x\in A:\, a-\epsilon < x \leq a$.\\\\
For $b= g.l.b. (A)$, the statement becomes $$\forall\epsilon\in\Breal, \exists\, x\in A:\, b\leq  x < b+\epsilon$$
Suppose for a given $\epsilon$,  $\not\exists\, x\in A:\, x < b+\epsilon$, then, $x \leq b+\epsilon$. Thus $b+\epsilon$ is a upper bound which is greater than $g.l.b.(S)$. We have a contradiction and thus $\forall\epsilon\in\Breal, \exists\, x\in A:\, b\leq  x < b+\epsilon$.
$$\blacklozenge$$

\subsection{}
%\subsubsection{}{}
\begin{tcolorbox}

Prove that if $A$ is a nonempty bounded set of real numbers then $g.l.b. (A) \leq l.u.b. (A)$. For what kind of set $A$ is $g.l.b. (A) =l.u.b. (A)$? 
\end{tcolorbox}
$A$ is bounded, hence by the upper bound property and its corollary (see Exercise 2.30.2) then $A$ has both a $l.u.b.(A)$  and a $g.l.b.(A)$. This mean that for every $x\in A$ we have $l.u.b.(A)\leq x\leq g.l.b.(A)$ and thus $l.u.b.(A)\leq g.l.b.(A)$\\\\
We can have $g.l.b. (A) =l.u.b. (A)$ when $A$ is a singleton, i.e. $A=\{x\}$. E.g. $A=f[\Breal]$ with $f=\{c\}$ with $c$ a constant, has $g.l.b. (A) =l.u.b. (A)=c$ .
$$\blacklozenge$$



\subsection{}
%\subsubsection{}{}
\begin{tcolorbox}
Prove that if $A$ and $B$ are nonempty subsets of $\Breal$ and $A \subset B$, then $g.l.b. (B) \leq g.l.b. (A)\leq  l.u.b. (A) \leq l.u.b. (B)$. 
\end{tcolorbox}
From $\mathbf{2.30.8}$ we know that $l.u.b.(A)\leq g.l.b.(A)$ and $l.u.b.(B)\leq g.l.b.(B)$. \\\\
Suppose we have $g.l.b.(A)< g.l.b.(B)$. This implies that $\exists x\in A: x\not \in B$. We have a contradiction as $A\subset B$ and thus we must have $g.l.b.(B)\leq g.l.b.(A)$.\\
In the same vein, suppose we have $l.u.b.(B)< l.u.b.(A)$. This implies that $\exists x\in A: x\not \in B$. We have a contradiction as $A\subset B$ and thus we must have $l.u.b.(A)\leq l.u.b.(B)$. \\
Putting this all together we get indeed
$$g.l.b. (B) \leq g.l.b. (A)\leq  l.u.b. (A) \leq l.u.b. (B)$$
$$\blacklozenge$$
\newpage

\mysection{31}{ Distance between two points in $\Breal$}

\renewcommand{\thesubsection}{\thesection.\arabic{subsection}}
\setcounter{subsection}{0}
\subsection{}
%\subsubsection{}{}
\begin{tcolorbox}
Verify the properties stated in $\mathbf{31.2}$
\end{tcolorbox}
$\mathbf{31.2(a)}$ and $\mathbf{31.2(b)}$ are trivial as $d(x,y)=|x-y|$.\\\\
$\mathbf{31.2(c)}$:\\
We consider 4 possibilities (the cases $a=b$ or $b=c$ are excluded as in that case $d(a,b)$ or $d(b,c)$ are zero).\\
i) $a<b$ and $b<c$. Then, 
$$d(a,c)= |a-b+b-c| = |-|a-b|-|b-c||= | |a-b|+|b-c| |= d(a,b)+d(b,c)$$
ii) $a<b$ and $b>c$. Then, 
$$d(a,c)= |a-b+b-c| = |-|a-b|+|b-c||< | |a-b|+|b-c| |= d(a,b)+d(b,c)$$
ii) $a>b$ and $b<c$. Then, 
$$d(a,c)= |a-b+b-c| = ||a-b|-|b-c||< | |a-b|+|b-c| |= d(a,b)+d(b,c)$$
iv) $a>b$ and $b>c$. Then, 
$$d(a,c)= |a-b+b-c| = ||a-b|+|b-c||= | |a-b|+|b-c| |= d(a,b)+d(b,c)$$
and conclude $$d(a,b)+d(b,c) \geq d(a,c)$$
$$\blacklozenge$$

\subsection{}
%\subsubsection{}{}
\begin{tcolorbox}
Let $p\in\Breal$. Give an example of a collection $\mathscr{K}$ of neighborhoods of $p$ such that $\bigcap\mathscr{K}$ is not a neighborhood. Show that if $\mathscr{K}$ is a nonempty finite collection of neighborhoods of $p$, then $\bigcup\mathscr{K}$ and $\bigcap\mathscr{K}$ are neighborhoods of $p$.
\end{tcolorbox}
Consider the infinite collection $\mathscr{K}=\{N(p;\frac{1}{n}):n\in \Pee\}$, then $\bigcap\mathscr{K}=\{p\}$ and no $\epsilon>0$ can be found for which $N(p;\epsilon)$ exists. \\\\
If $\mathscr{K}$ is a finite collection then $\mathscr{K}$ is countable with element $N(p;\epsilon_i)$ and we can construct a finite set $$\mathscr{E}=\{\epsilon_i:i=1,2,\dots,n\}$$  As $ \mathscr{E}\subset \Breal$ and by the least upper bound principle, $\mathscr{E}$ has both a $\epsilon_u= l.u.b.(\mathscr{E})$ and a $\epsilon_l = g.l.b(\mathscr{E})$. Moreover, $N(p;\epsilon_u)$ and  $N(p;\epsilon_l)$ are elements of $\mathscr{K}$ and hence we get\\\\
$$\bigcup\mathscr{K}=N(p;\epsilon_u)$$ and $$\bigcap\mathscr{K}=N(p;\epsilon_l)$$.
$$\blacklozenge$$
\newpage


\mysection{32}{Limit of a sequence in $\Breal$}

\renewcommand{\thesubsection}{\thesection.\arabic{subsection}}
\setcounter{subsection}{0}
\subsection{}
%\subsubsection{}{}
\begin{tcolorbox}
Suppose that $(a_n)$ is a sequence such that $a_n\leq a_{n+1}$, $(a_{n+1}\leq a_{n})$ for each positive integer $n$. Suppose further that the sequence $(a_n)$ is bounded above (below). Then $\lim (a_n)$ exists. 
\end{tcolorbox}
$$ $$ 
$\mathbf{a_n\leq a_{n+1}}$\\\\
As $(a_n)$ is bounded above and is a subset of $\Breal$, then by the upper bound property of $\Breal$, has a lower upper bound, $A=l.u.b.$.\\
We first notice that for any arbitrary $\epsilon\in \Breal$, there must be a $a_N$ such that $A-\epsilon < a_N\leq A$ (as otherwise $A-\epsilon$ would be an upper bound, giving a contradiction).\\
So we have $A-\epsilon < a_N$ and thus $A-a_N < \epsilon$.  Furthermore as $a_{N+k}\geq a_N$ for all $k$ we will have $A-a_{N+k}\leq A-a_N<\epsilon$.  As $\forall n, A\geq a_n $, $A-a_n$ will stay positive for all $n$. Thus, the last result can be written as $|A-a_{N+k}|<\epsilon$ for a chosen $\epsilon$. From the definition of the limit, we conclude that the $\lim (a_n)= l.u.b.(a_n)$ exists.\\\\
$\mathbf{a_{n+1}\leq a_{n}}$\\\\
As $(a_n)$ is bounded below  and is a subset of $\Breal$, then by the corollary of the upper bound property of $\Breal$, has a greatest lower bound, $L=g.l.b.$.\\
We first notice that for any arbitrary $\epsilon\in \Breal$, there must be a $a_N$ such that $L\leq a_N < L+\epsilon$ (as otherwise $L+\epsilon$ would be a lower bound, giving a contradiction).\\
So we have $a_N < L+\epsilon$ and thus $a_N -L < \epsilon$.  Furthermore as $a_{N+k}\leq a_N$ for all $k$ we will have $a_{N+k}-L\leq a_{N}-L<\epsilon$.  As $\forall n, L\leq a_n $, $a_n-L$ will stay positive for all $n$. Thus, the last result can be written as $|L-a_{N+k}|<\epsilon$ for a chosen $\epsilon$. From the definition of the limit, we conclude that the $\lim (a_n)= g.l.b.(a_n)$ exists.
$$\blacklozenge$$

\subsection{}
%\subsubsection{}{}
\begin{tcolorbox}
Let the sequence $(c_n)$ in $\Breal$ be given by $c_n = a_n+b_n$, where $\lim(a_n)= A$ and $\lim(b_n)= B$.\\
 Then, $\lim (c_n)= A +B$.
\end{tcolorbox}
From the definition of the limit of a sequence, for a given $\frac{\epsilon}{2}$ there will be elements $a_N$ and $b_{N^{'}}$ such that for all $n\geq N$ and $n^{'}\geq {N^{'}}$ we will have 
$$|A-a_n|<\frac{\epsilon}{2}\Et |B-b_{n^{'}}|<\frac{\epsilon}{2}$$
Let's put $M=sup(N,\,N^{'})$. 
Adding the two inequalities gives 
$$|A-a_m|+|B-b_{m^{}}|<\epsilon$$
for any $m\geq M$.
The triangle inequality of the distance in $\Breal$ states
$$d(x,z)\leq d(x,y)+d(y,z)$$
Put $x= a_m-B,\, z= A-b_m$ and $y= A-B$. The triangle inequality gives $|a_m-B-A+b_m  | \leq |a_m-B-A+B |+|A-b_m -A+B |$ or
$$|a_m+b_m-(A+B)| \leq |A-a_m |+|B-b_m |$$
But $|A-a_m|+|B-b_{m^{}}|<\epsilon$ and get 
$$|a_m+b_m-(A+B)| <\epsilon$$
giving $\lim (a_n+b_n)= A+B$
$$\blacklozenge$$

\subsection{}
%\subsubsection{}{}
\begin{tcolorbox}
 Let the sequence $(c_n)$ in $\Breal$ be given by $c_n = ka_n$, where $k$ is a constant and $\lim (a_n)= A$.\\
 Then, $\lim (c_n)= kA$.
\end{tcolorbox}
Be given $k$. Let's chose a $\frac{\epsilon}{|k|}$ (we suppose $k\neq 0$, this case being trivial).\\
We have $|a_n-A|<\frac{\epsilon}{|k|}$ for all $n\geq$ then  certain $N$. Multiplying by $|k|$ gives $|k||a_n-A|<\frac{|k|\epsilon}{|k|}$ and thus
$$|ka_n-kA|<\epsilon$$
and conclude $\lim ka_n= kA$.
$$\blacklozenge$$
\newpage
\subsection{}
%\subsubsection{}{}
\begin{tcolorbox}
 Let the sequence $(c_n)$ in $\Breal$ be given by $c_n = a_nb_n$, where $\lim (a_n)= A$ and $\lim (b_n)= B$.\\
 Then, $\lim (c_n)= AB$. 
\end{tcolorbox}

Choose arbitrary $\epsilon_a,\,\epsilon_b>  0$. Then, \\ There exists a $n_a \in \Pee$, such that \begin{align}|a_n - A| < \epsilon_a \,\,\,\ \forall n\in\Pee, n<n_a\end{align}
there also exists a $n_b \in \Pee$, such that  \begin{align}|b_n - B| <\epsilon_b \,\,\,\ \forall n\in\Pee, n< n_b\end{align}
Also, there exists  a $\hat{n}$ such that
$$ |a_n|-|A| \leq ||a_n|-|A||\leq |a_n-A|<1$$
for all $n > \hat{n}$, so that $|a_n|<|A|+1$, i.e. $\frac{|a_n|}{|A|+1}<1$.\\
 Hence, for all  $n > \sup\{n_a,n_b,\hat{n}\}$ we have 
$$|a_nb_n - AB| = |a_nb_n -a_nB+a_nB-AB| = |a_n(b_n-B)+B(a_n-A)|$$ 
from which follows, using $(1)$ and $(2)$ 
\begin{align} |a_nb_n - AB|\leq |a_n(b_n-B)|+|B(a_n-A)| < |a_n|\epsilon_b+|B|\epsilon_a\end{align}
Let's put $\epsilon_a= \frac{\epsilon}{2(|B|+1)}$ and $\epsilon_b= \frac{\epsilon}{2(|A|+1)}$ such that we have
\begin{align}|a_n - A| < \frac{\epsilon}{2(|B|+1)} \,\,\,\ \forall n\in\Pee, n<n_a\\
|b_n - B| < \frac{\epsilon}{2(|A|+1)} \,\,\,\ \forall n\in\Pee, n< n_b
\end{align}
As $\frac{|a_n|}{|A|+1}<1$ (see above) and $\frac{|B|}{|B|+1}\leq 1$, we get for $(3)$, using $(4)$ and $(5)$   $$ |a_nb_n - AB|< \underbrace{\frac{|a_n|}{(|A|+1)}}_{<1}\frac{\epsilon}{2}+\underbrace{\frac{|B|}{(|B|+1)}}_{<1}\frac{\epsilon}{2}<\frac{\epsilon}{2} +\frac{\epsilon}{2} = \epsilon.$$
for any arbitrary $\epsilon$.\\\\\\
$$\blacklozenge$$

\subsection{}
%\subsubsection{}{}
\begin{tcolorbox}
Let the sequence $(c_n)$ in $\Breal$ be given by $c_n = \frac{a_n}{b_n}$, where $b_n\neq 0$,  $\lim (a_n)= A$ and $\lim (b_n)= B\neq 0$.\\
 Then, $\lim (c_n)= \frac{A}{B}$.
\end{tcolorbox}
Choose arbitrary $\epsilon_a,\,\epsilon_b>  0$. Then, \\ There exists a $n_a \in \Pee$, such that \begin{align}|a_n - A| < \epsilon_a \,\,\,\ \forall n\in\Pee, n<n_a\end{align}
there also exists a $n_b \in \Pee$, such that  \begin{align}|b_n - B| <\epsilon_b \,\,\,\ \forall n\in\Pee, n< n_b\end{align}
Also, there exists  a $\hat{n}$ such that
$$ |b_n|> \frac{|B|}{2} $$
for all $n > \hat{n}$ (as otherwise we would have $|b_n|\leq \frac{|B|}{2}$ for all $n > \hat{n}$, meaning that $|b_n|$ will not be able to get arbitrarily close to $|B|$.)\\
Thus we can write 
\begin{align}
\frac{1}{|b_n|}< \frac{2}{|B|}\quad \forall n > \hat{n}
\end{align}.\\\\
 Hence, for all  $n > \sup\{n_a,n_b,\hat{n}\}$ we have 
$$\left|\frac{a_n}{b_n} - \frac{A}{B}\right| = \left|\frac{a_n}{b_n}-\frac{A}{b_n}+\frac{A}{b_n} - \frac{A}{B}\right|\leq\frac{|a_n-A|}{|b_n|} +|A|\left|\frac{1}{b_n}-\frac{1}{B} \right|=\frac{|a_n-A|}{|b_n|} +|A|\frac{|B-b_n|}{|B| |b_n|}$$ Using $(1),\, (2)$ and $(3)$ we get 
\begin{align}
\left|\frac{a_n}{b_n} - \frac{A}{B}\right|&<  \frac{1}{|b_n|}\left(\epsilon_a + \frac{|A|}{|B|}\epsilon_b\right)\\
&<    \frac{2}{|B|}\left(\epsilon_a + \frac{|A|}{|B|}\epsilon_b\right)
 \end{align}
 Let's put $\epsilon_a = \frac{|B|}{4}\epsilon$ and $\epsilon_b = \frac{|B|^2}{4|A|}\epsilon$  we get 
\begin{align*}
\left|\frac{a_n}{b_n} - \frac{A}{B}\right|&<  \frac{1}{2}\epsilon+ \frac{1}{2}\epsilon\\
&=\epsilon 
\end{align*}
for any arbitrary $\epsilon$.
$$\blacklozenge$$
\newpage
\subsection{}
%\subsubsection{}{}
\begin{tcolorbox}
If a sequence $(a_i)$ has a limit, it is unique.
\end{tcolorbox}
Suppose we have for an arbitrary $\epsilon$
\begin{align*}
&\exists n_1,\, |a_n-A|<\epsilon,\, \forall n>n_1\\
&\exists n_2,\, |a_n-A^{'}|<\epsilon,\, \forall n>n_2
\end{align*}

Suppose that $A \neq A^{'}$. Let $\epsilon = \frac{|A - A^{'}|}{2} $. By hypothesis there exists a $n_1 \in \Pee$ such that 
$$
|a_n - A| <\frac{|A - A^{'}|}{2} \quad \text{if} \quad n \geq n_1 
$$ 
By hypothesis, there exists $n_2 \in \Pee$ such that 
$$
|a_n - A^{'}|<\frac{|A - A^{'}|}{2} \quad \text{if} \quad n \geq n_2 
$$
Let $\hat{n} = sup\{n_1,n_2\}$. If $n \geq \hat{n}$, then by the triangle inequality
$$
|A - A^{'}| = |(a_n - A) - (a_n - a^{'})| <|a_ n - A| + |a_n - A^{'}| < 2 \frac{|A - A^{'}|}{2} = |A - A^{'}|
$$
And we have a contradiction as we get $
|A - A^{'}| <|A - A^{'}|
$.
$$\blacklozenge$$
\newpage

\mysection{33}{The nested interval theorem for $\Breal$}

\renewcommand{\thesubsection}{\thesection.\arabic{subsection}}
\setcounter{subsection}{0}
\subsection{}
%\subsubsection{}{}
\begin{tcolorbox}
Give the details of the proof of Theorem $\mathbf{33.1}$.
\end{tcolorbox}
First we note that we have 
\begin{align}
a_1\leq \dots \leq a_n\leq a_{n+1}\leq\dots \leq b_1\\
a_1\leq \dots \leq b_{n+1}\leq b_n \leq\dots \leq b_1
\end{align}
We see that the subset $\{a_n\}\subset\Breal $ is bounded above, and the subset $\{b_n\}\subset\Breal $ is bounded below, hence, by the upper bound principle (and its corollary) in $\Breal$, $\{a_n\}$ has a $l.u.b.$ and  $\{b_n\}$ has a $g.l.b.$ Let's denote them $A$ and $B$ respectively.\\
Be $\epsilon>0$, then there exists a $n_1\in\Pee$ such that  $\forall n\geq n_1 $ we have $A-\epsilon < a_n\leq A$ (as otherwise $A-\epsilon$ would be an upperbound, which is contradictory with the fact that $A$ is a $l.u.b.$
Then, we have as $\epsilon>0$, 
$$A-\epsilon < a_n< A+\epsilon$$ and thus 
\begin{align}|a_n-A|<\epsilon\end{align}
Giving $A=\lim \{a_n\}$\\\\
The same reasoning on the lower bound of $b_n$ gives $B\leq b_n < B+\epsilon$ and thus 
\begin{align}|b_n-B|<\epsilon\end{align}
and $B=\lim \{b_n\}$\\\\
Then we have 
\begin{align}
a_1\leq \dots \leq a_n\leq a_{n+1}\leq\dots \leq A\leq b_1\\
a_1\leq B \leq \dots \leq b_{n+1}\leq b_n \leq\dots \leq b_1
\end{align}
From this we conclude that $A\leq B$. Indeed, suppose we would have $A>B$.\\
From $(5)$ and $(6)$ we can write $(3)$ and $(4)$ as 
\begin{align}
|a_n-A|= A-a_n<\epsilon\quad \forall n\geq n_1\\
|b_n-B|= b_n-B<\epsilon\quad \forall n\geq n_2
\end{align}
Choose $\epsilon = \frac{A-B}{2},\, (>0$ as we supposed $A>B)$ and  $\hat{n}=\max(n_1,\, n_2)$. Adding $(7)$ and $(8)$ gives
\begin{align}
&A-B+ b_n-a_n<\epsilon+\epsilon\\
\implies\quad &A-B+ b_n-a_n <A-B\\
\implies\quad & b_n <a_n
\end{align}
which is impossible as we have $a_n \leq b_n,\, \forall n\in \Pee$.\\\\
Till now, we proved that $A=\lim \{a_n\} $ and $B=\lim \{b_n\}$ exist and that $A\leq B$.\\\\
Consider $[A,\, B]$ and an element $x\in [A,\, B]$. We have 
$$A\leq x\leq B$$
But we know $a_n\leq A\Et b_n\geq B,\, \forall n\in \Pee$ and so we get  
$$a_n\leq A\leq x\leq B\leq b_n$$
and thus $x\in [a_n,b_n]$ and conclude 
$$[A,B]\subset [a_n,b_n],\,  \forall n \in \Pee$$ and we can state
$$\exists x\in  \bigcap^{\infty}_{i=1}[a_i,b_i]$$ and thus
$$\bigcap^{\infty}_{i=1}[a_i,b_i]\neq \emptyset$$
\\\\
We now prove that if $\lim \left(|b_i -a_i|\right)=0$ then $\bigcap^{\infty}_{i=1}[a_i,b_i]$ has exactly one element.\\
Suppose $x,\, y\in \bigcap^{\infty}_{i=1}[a_i,b_i]$, two different elements. Then $| b_i-a_i| \geq |x-y| >0,\, \forall i\in \Pee$ as $x\neq y$. Choose $\epsilon=\frac{|x-y|}{2}$, then 
$| b_i-a_i|  >\epsilon ,\, \forall i\in \Pee$ and thus $|b_i-a_i|$ doe not have a limit. We get a contradiction as we supposed $\lim \left(|b_i -a_i|\right)=0$.
$$\blacklozenge$$
\newpage
\subsection{}
%\subsubsection{}{}
\begin{tcolorbox}
Give an example of nonempty intervals $I_i$ (see $\mathbf{33.2}$) such that 
$$I_1\supset I_2 \supset \dots \supset I_n\supset \dots  \Et \bigcap^{\infty}_{i=1}I_i=\emptyset $$
\end{tcolorbox}
Choose $I_i=(0,\, 2^{-i})$. Suppose there exists $x\in \bigcap^{\infty}_{i=1}I_i$, then for any $i\in \Pee$, there exists a $n > i$ such that $x\geq 2^{-n}$ and thus $x\not\in I_n$ and hence $x\not\in \bigcap^{\infty}_{i=1}I_i$.\\\\
Thus $$\bigcap^{\infty}_{i=1}I_i=\emptyset $$

$$\blacklozenge$$

\subsection{}
%\subsubsection{}{}
\begin{tcolorbox}
Is the following statement true? Given an interval $I$ and a point $p\in I$, there exists a countable collection of closed intervals $\{I_i\}$ such that $p\in I_1\subset I_2\subset I_3\dots \subset I_n\dots $ (possibly only a finite number needed) and such that $I=\bigcup^{\infty}_{i=1}I_i$.
\end{tcolorbox}
Without loss of generalization, put $I=(0,1)$ and $p\in I$. Define, the following collection of closed sets
\begin{align*}
I_n=\left[a_n,\, b_n\right]=\left[\frac{p}{2^n},\, \frac{p+2^n-1}{2^n}\right]
\end{align*}
We have $I_n\subset I_{n+1}$. Indeed,
\begin{align*}
b_{n+1}=\frac{p+2^{n+1}-1}{2^{n+1}}&=\half \frac{p+2.2^{n}-1}{2^{n}}\\
&= \half \frac{2p+2.2^{n}-2+1-p}{2^{n}}\\
&=  \frac{p+2^{n}-1}{2^{n}}+\half \underbrace{\frac{1-p}{2^{n}}}_{>0}\\
&>\frac{p+2^{n}-1}{2^{n}}=b_n
\end{align*}
and also $a_{n+1}=\frac{p}{2^{n+1}}<\frac{p}{2^n}=a_n$ giving $I_n\subset I_{n+1}$.\\\\
Also, for every $n\in\Pee $ we have $a_n=\frac{p}{2^n}>0$ and 
\begin{align*}
b_n=\frac{p+2^{n}-1}{2^{n}}&= \underbrace{\frac{p-1}{2^{n}}}_{<0}+1\\
&< 1
\end{align*}
Moreover $a_n< p,\, \forall p\in \Pee$ and 
\begin{align*}
b_n-p&=\frac{p+2^{n}-1}{2^{n}}-p\\
&= \frac{p+2^{n}-1-2^np}{2^{n}}\\
&= \frac{(2^{n}-1)-(2^n-1)p}{2^{n}}\\
&= \underbrace{\frac{(2^{n}-1)}{2^{n}}}_{>0}\underbrace{(1-p)}_{>0}\\
&>0\\
\implies \quad b_n&>p
\end{align*}
So $I_n$  are closed intervals containing $p$ and for which we have,
$$p\in I_1\subset I_2\subset I_3\dots \subset I_n\dots $$
and 
$$I_n\subset I,\, \forall n\in \Pee$$\\
Obviously, the collection of $I_n$ is countable as $n\in \Pee$ and $\Pee$ is countable.
\\\\
We now prove that 
$$I=\bigcup^{\infty}_{i=1}I_i $$
Suppose we have $x\in I$, then, $$\exists \hat{n}\in \Pee,\, \frac{p}{2^{\hat{n}}}\leq x\leq \frac{p+2^{\hat{n}}-1}{2^{\hat{n}}}$$
Indeed, it suffice to take $\hat{n}= max(\log_2 \frac{p}{x},\, \log_2 \frac{1-p}{1-x})$ with $\hat{n}\in\Pee$.\\
So, for every $x\in I$ we can find a closed interval containing $x$ and thus $$I\subset  \bigcup^{\infty}_{i=1}I_i$$
and as $I_i\subset I,\, \forall i\in \Pee$ we have also 
$$\bigcup^{\infty}_{i=1}I_i\subset  I$$
giving
$$I=\bigcup^{\infty}_{i=1}I_i$$
$$\blacklozenge$$


\subsection{}
%\subsubsection{}{}
\begin{tcolorbox}
Suppose that $\mathscr{K}$ is a collection of intervals such that $\bigcap\mathscr{K}\neq \emptyset$. Is $\bigcup\mathscr{K}$ necessarily an interval?
\end{tcolorbox}
Be $a\in \bigcup \mathscr{K}\Et b\in \bigcup \mathscr{K}$. Consider the closed interval $[a,\, b]$. Is it possible that $[a,\, b]\not \subset \bigcup \mathscr{K}$?\\\\
$[a,\, b]\not \subset \bigcup \mathscr{K}$ would mean that $\exists x\in [a,\, b]$ such that $x\not\in \bigcup \mathscr{K}$.\\
As $a\in \bigcup \mathscr{K}$ there exists at least one interval $K_a\in \mathscr{K}$ such that $a\in K_a$. In the same way $\exists K_b\in\mathscr{K}$ for which we have $b\in K_b$.\\\\
If we have $a\in K_a$ and also $b\in K_a$, then we can put $K_a=K_b$ with the consequence that  $x\in K_a$ as $K_a$ is an interval and hence $\forall a,\, b\in K_a, [a,\, b]\subset K_a$.\\
Hence, $x$ will also be an element of $\bigcup \mathscr{K}$ and we get $[a,\, b]\subset \bigcup \mathscr{K}$.\\\\
Suppose now, $K_a\neq K_b$. As we suppose $x\in [a,\, b]$, but $x\not\in K_a$ and $x\not\in K_b$ we will have 
$$a<x<b\quad(\text{compared to } a\leq x\leq b)$$
As $a$ and $b$ are arbitrary (insofar $a\in K_a$ and $b\in K_b$) and $x\not \in K_a\cup K_b$, we get $$K_a\cap K_b=\emptyset$$
This is not possible as we have $\bigcap\mathscr{K}\neq \emptyset$ and thus $x$ must be an element of $\bigcup \mathscr{K}$ implying $[a,\, b]\subset \bigcup \mathscr{K}$.\\\\
\textbf{Conclusion}: Provided that we have $\bigcap\mathscr{K}\neq \emptyset$ , $\bigcup \mathscr{K}$ is indeed an interval 
$$\blacklozenge$$
\newpage

\mysection{34}{Algebraic structure for  $\Breal^{n}$}

\renewcommand{\thesubsection}{\thesection.\arabic{subsection}}
\setcounter{subsection}{0}

\subsection{}
%\subsubsection{}{}
\begin{tcolorbox}
Verify the properties stated in $\mathbf{34.2}$ and $\mathbf{34.4}$.
\end{tcolorbox}

$$\blacklozenge$$\\


\subsection{}
\begin{tcolorbox}
Show that for points in $\Breal^2$, $|x-y|$ gives the usual distance formula with which the reader is familiar from analytic geometry.
\end{tcolorbox}

$$\blacklozenge$$\\
\subsection{}
\begin{tcolorbox}
Let $x,\, y,\, \Et z$ be distinct points in $R^2$. Let $L_1$ be the line segment with endpoints $x$ and $z$. Let $L_2$ be the line segment with endpoints $y$ and $z$. Let $\alpha$ be the smaller angle (or one of the angles if equal) formed by $L_1$ and $L_2$ at $z$. Show that $$\cos\alpha = \frac{(x-z)(y-z)}{|x-z| |(y-z|}$$
\end{tcolorbox}

$$\blacklozenge$$\\

\subsection{}
\begin{tcolorbox}
Let $x$ and $y$ be two elements in $R^2$. Show that $|x. y|\leq |x||y|$ and $|x+y|\leq |x|+|y|$
\end{tcolorbox}

$$\blacklozenge$$\\
\subsection{}
\begin{tcolorbox}
The results of this exercise will be needed in the next section. Consider the function $f$ given by $f (x) = A x^2+ 2Bx+ C$, where $A>0$, and which further satisfies $f(x) \geq 0$ for all real $x$. Prove that $B^2-AC\leq 0$. 
\end{tcolorbox}

$$\blacklozenge$$
\newpage