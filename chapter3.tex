\chapter{Metric Spaces: Introduction}
\pagebreak[4]

\mysection{45}{Distance function and metric spaces}
\renewcommand{\thesubsection}{\thesection.\arabic{subsection}}
\setcounter{subsection}{0}
\subsection{}
\begin{tcolorbox}
Verify that the functions given in $45.2$ through $45.6$ are metrics.
\end{tcolorbox}

$\mathbf{45.2.}$ Let $k > 0$ and $\rho :\Breal^n\times\Breal^n\rightarrow \Breal$ be given by
$$\rho(x, y) = k |x-y| \text{ for all } x \Et y \text{ in }\Breal^n$$
$$\triangledown$$
\begin{align*}
\begin{array}{l l l l l}
 45.1(a).& d(x, y) = 0 \Iff  x =y&:&\rho(x, x)& = k |x-x| = 0\\
 45.1(b). &d(x, y) = d(y, x)&:&\rho(x, y)& = k |x-y| = k |y-x|=\rho(y, x)\\
45.1 (c). &d(x,y) + d(y, z)\geq d(x, z)&:&\rho(x, z)&= k |x-y|\\
& & & &= k |x-z + z-y|\\
& & & &\leq k |x-z| + k|z-y|\\
& & & &= \rho(x, y) + \rho(y, z)
\end{array}
\end{align*}
$$\lozenge$$
$\mathbf{45.3.}$ The function $g:\Breal^2\times\Breal^2\rightarrow\Breal$ given by
$$g(x, y) = |x_1 -y_1| + |x_2 -y_2|$$
where $x=(x_1,x_2)$ and $y=(y_1,y_2)$

$$\triangledown$$
\begin{align*}
\begin{array}{l l l l l}
 45.1(a).& d(x, y) = 0 \Iff  x =y&:&g(x, x)& =|x_1 -x_1| + |x_2 -x_2| = 0\\
 45.1(b). &d(x, y) = d(y, x)&:&g(x, y)&=|x_1 -y_1| + |x_2 -y_2|\\
 &&&&=|y_1 -x_1| + |y_2 -x_2|\\
  &&&&=g(y,x)\\
45.1 (c). &d(x,y) + d(y, z)\geq d(x, z)&:&g(x, z)&=|x_1 -z_1| + |x_2 -z_2|\\
 &&&&=|x_1 -y_1+y_1-z_1| + |x_2-y_2+y_2 -z_2|\\
 &&&&\leq|x_1 -y_1|+|y_1-z_1| + |x_2-y_2|+|y_2 -z_2|\\
 &&&&=|x_1 -y_1|+ |x_2-y_2|+|y_1-z_1| +|y_2 -z_2|\\
 &&&&=g(x,y)+g(y,z)
\end{array}
\end{align*}
$$\lozenge$$
$\mathbf{45.4.}$  Let $X$ be an arbitrary set. Define $m: X \times X \rightarrow \Breal$ as follows:
\begin{align*}
m(x,y) = 0 \text{ if } x=y\\
m(x,y)=1\text{ if } x\neq y 
\end{align*}

$$\triangledown$$
\begin{align*}
\begin{array}{l l l l l}
 45.1(a).& d(x, y) = 0 \Iff  x =y&:&m(x, x)& = 0\text{ (by definition)}\\
 45.1(b). &d(x, y) = d(y, x)&:&m(x, y)&=1\\
 &&&&=m(x, y)\text{ (by definition)}\\
45.1 (c). &d(x,y) + d(y, z)\geq d(x, z)&:&m(x, z)&=1 \text{ if } x\neq z\\
&&&&\leq 1+1\\
&&&&= m(x, y)+m(y, z)\text{ if } y\neq x\Et y\neq z\\
&&&m(x, z)&=1 \text{ if } x\neq z\\
&&&&= 1+0\\
&&&&= m(x, y)+m(y, z)\text{ if } y\neq x\Et y= z\\
&&&m(x, z)&=0 \text{ if } x= z\\
&&&&\leq 1+1\\
&&&&= m(x, y)+m(y, z)\text{ if } y\neq x\Et y\neq z\\
&&&m(x, z)&=0 \text{ if } x= z\\
&&&&\leq 1+0\\
&&&&= m(x, y)+m(y, z)\text{ if } y\neq x\Et y=z
\end{array}
\end{align*}
$$\lozenge$$
$\mathbf{45.5.}$ Let $p$ be the real-valued function defined on $\Breal^2\times\Breal^2$ by 
$$p(x,y)= \max{\{|x_i-y_i|: i=1,2\}}$$
where $x=(x_1,x_2)$ and $y=(y_1,y_2)$. The function $p$ is a metric for $\Breal^2$.

$$\triangledown$$
\begin{align*}
\begin{array}{l l l l l}
 45.1(a).& d(x, y) = 0 \Iff  x =y&:&p(x, x)& =\max{\{|x_i-x_i|: i=1,2\}}\\
 &&&&=\max{\{0,\, 0\}}\\
 &&&&=0\\
 45.1(b). &d(x, y) = d(y, x)&:&p(x, y)&=\max{\{|x_i-y_i|: i=1,2\}}\\
 &&&&=\max{\{|y_i-x_i|: i=1,2\}}\\
  &&&&=p(y,x)\\
45.1 (c). &d(x,y) + d(y, z)\geq d(x, z)&:&p(x, z)&=\max{\{|x_i-z_i|: i=1,2\}}\\
&&&&=\max{\{|x_i-y_i+y_i-z_i|: i=1,2\}}\\
&&&&\leq\max{\{|x_i-y_i|+|y_i-z_i|: i=1,2\}}\\
&&&&\leq\max{\{\max{\{|x_i-y_i|\}}+\max{\{|y_i-z_i|\}}: i=1,2\}}\\
&&&&=\max{\{|x_i-y_i|: i=1,2\}}+\max{\{|y_i-z_i|: i=1,2\}}\\
&&&&= p(x,y)+p(y,z)
\end{array}
\end{align*}
$$\lozenge$$
$\mathbf{45.6.}$ Let the function $h$ be defined as follows: For all $x$ and $y$ in $\Breal^2$, let
$$h(x,y)=\frac{|x-y|}{1+|x-y|}$$
The function is a metric for $\Breal^2$
$$\triangledown$$
\begin{align*}
\begin{array}{l l l l l}
 45.1(a).& d(x, y) = 0 \Iff  x =y&:&h(x, x)& =\frac{|x-x|}{1+|x-x|}\\
 &&&&=0\\
 45.1(b). &d(x, y) = d(y, x)&:&h(x, y)&=\frac{|x-y|}{1+|x-y|}\\
 &&&&=\frac{|y-x|}{1+|y-x|}\\
 &&&&= h(y,x)\\
45.1 (c). &d(x,y) + d(y, z)\geq d(x, z)&:&h(x, z)&=\frac{|x-z|}{1+|x-z|}
\end{array}
\end{align*}
For this last requirement, consider the function  $g(t)= \frac{t}{1+t},\, t\geq 0$. Its derivative is $g{'}(t)= \frac{t}{(1+t)^2},\, t\geq 0$. As $g{'}(t)>0,\, t>0$, $g(t)$ is a monotone increasing function and thus, using this fact in our function $h(x,y)$ we find that
\begin{align*}
h(x, z)&=\frac{|x-z|}{1+|x-z|}\\
&=\frac{|x-y+y-z|}{1+|x-y+y-z|}\\
&\leq\frac{|x-y|+|y-z|}{1+|x-y|+|y-z|}\quad (\text{ because }|x-y|+|y-z|\geq |x-z|)\\
&=\frac{|x-y|}{1+|x-y|+|y-z|}+\frac{|y-z|}{1+|x-y|+|y-z|}\\
&\leq \frac{|x-y|}{1+|x-y|}+\frac{|y-z|}{1+|y-z|}\\
&=h(x,y)+h(y,z)
\end{align*}
$$\blacklozenge$$

\renewcommand{\thesubsection}{\thesection.\arabic{subsection}}
%\setcounter{subsection}{0}
\subsection{}
\begin{tcolorbox}
Let $(X, d)$ be a metric space. Define $d^*:X \times X \rightarrow \Breal$ as follows:
$$d^*(x,y) = \min{\{1, d(x, y)\}}$$
for all $x$ and $y$ in $X$. Show that $d^*$ is a metric for $X$.
\end{tcolorbox}
\begin{align*}
\begin{array}{l l l l l}
 45.1(a).& d(x, y) = 0 \Iff  x =y&:&d^*(x, x)& =\min{\{1, d(x, x)\}}\\
 &&&&=\min{\{1, 0\}}\\
 &&&&=0\\
 45.1(b). &d(x, y) = d(y, x)&:&d^*(y, x)&=\min{\{1, d(x, y)\}}\\
 &&&&=\min{\{1, d(y, x)\}}\\
 &&&&=d^*(x, y)\\
45.1 (c). &d(x,y) + d(y, z)\geq d(x, z)&:&d^*(x, z)&=\min{\{1, d(x, z)\}}
\end{array}
\end{align*}
For this last requirement, let's consider the following cases\\
i) $d(x, z)\leq 1$. Then we have $d^*(x, z)=d(x, z)$ and thus $d^*(x, z)\leq 1$.\\
 As $d(x, z)\leq d(x, y)+d(y, z)$ we have $d^*(x, z)\leq d(x, y)+d(y, z)$.\\
 Let's examine the following possibilities:
\begin{align*}
\begin{array}{l c  l}
d(x, y)\Et d(y, z)\geq 1&: &d^*(x, y)+d^*(y, z)=2>d^*(x, z)\\
d(x, y)\leq 1\Et d(y, z)\geq 1&: &d^*(x, y)+d^*(y, z)=1+d(x, y)\geq 1 \geq d^*(x, z)\\
d(x, y)\geq 1 \Et d(y, z)\leq 1&: &d^*(x, y)+d^*(y, z)=1+d(y, z)\geq 1\geq d^*(x, z)\\
d(x, y)\Et d(y, z)\leq 1&: &d^*(x, y)+d^*(y, z)=d(x, y)+d(y, z)\geq d^*(x, z)
\end{array}
\end{align*}
and can conclude:
$$d^*(x, z)\leq d^*(x, z)+d^*(x, z)$$
ii) $d(x, z)\geq 1$. Then we have $d^*(x, z)=1$. By the triangle inequality we have $d(x, z)\leq d(x, y)+d(y, z)$ and as $d(x, z)\geq 1$, we are sure that $d(x, y)+d(y, z)\geq 1$.\\
Let's examine the following possibilities:
\begin{align*}
\begin{array}{l c  l}
d(x, y)\Et d(y, z)\geq 1&: &d^*(x, y)+d^*(y, z)=2>1\\
d(x, y)\leq 1\Et d(y, z)\geq 1&: &d^*(x, y)+d^*(y, z)=1+d(x, y)\geq 1\\
d(x, y)\geq 1 \Et d(y, z)\leq 1&: &d^*(x, y)+d^*(y, z)=1+d(y, z)\geq 1
\\
d(x, y)\Et d(y, z)\leq 1&: &d^*(x, y)+d^*(y, z)=d(x, y)+d(y, z)\geq 1
\end{array}
\end{align*}
and get, as $d^*(x, z)=1$, $$d^*(x, z)\leq d^*(x, z)+d^*(x, z)$$
$$\blacklozenge$$

\renewcommand{\thesubsection}{\thesection.\arabic{subsection}}
%\setcounter{subsection}{0}
\subsection{}
\begin{tcolorbox}
Let $d^*:X \times X \rightarrow \Breal$ be a function that satisfies the following:
\begin{align*}
&d(x,y)=0 \text{ if and only if } x=y\\
&d(z,x)+d(z,y)\geq d(x,y)
\end{align*}
Show that for all $x$ and $y$ in $X$, $d(x,y)\geq 0$ and $d(x,y)=d(y,x)$. (Hence, a function satisfying the two given properties is a metric for $X$.)
\end{tcolorbox}
i) $d(x,y)\geq 0 \text{ for all  } x,\,y $:\\
From the second requirement we have
\begin{align*}
&d(y,y)\leq d(x,y)+d(x,y)\\
\implies\quad &0\leq 2d(x,y)\\
\implies\quad &d(x,y)\geq 0
\end{align*}
ii) $d(x,y)=d(y,x)\text{ for all  } x,\,y $:\\
We want to find two related expressions like $d(x,y)\leq d(y,x)$ and $d(y,x)\leq d(x,y)$.\\
From the second requirement we have
\begin{align*}
&\left\{\begin{array}{ll}
d(x,y)\leq d(z,x)+d(z,y)&(1)\\
d(y,x)\leq d(z,x)+d(z,y)&(2)
\end{array}\right.\\
&\left\{\begin{array}{ll}
d(x,z)\leq d(y,x)+d(y,z)&(3)\\
d(z,x)\leq d(y,x)+d(y,z)&(4)
\end{array}\right.\\&\left\{\begin{array}{ll}
d(y,z)\leq d(x,y)+d(x,z)&(5)\\
d(z,y)\leq d(x,y)+d(x,z)&(6)
\end{array}\right.\\
\end{align*}
from which we can get:
\begin{align*}
&\left\{\begin{array}{llc}
(4)\rightarrow (1): &d(x,y)\leq d(y,x)+ [d(y,z)+d(z,y)]&(7)\\
(6)\rightarrow (1):&d(y,x)\leq d(x,y)+ [d(x,z)+d(z,x)]&(8)
\end{array}\right.
\end{align*}
In $(7)$, put $z=y$ and in $(8)$ put $z=x$. We get, as $d(a,b)=0\text{ if } a=b$:
\begin{align*}
&\left\{\begin{array}{l}
d(x,y)\leq d(y,x)\\
d(y,x)\leq d(x,y)
\end{array}\right.
\end{align*}
from which we conclude that $d(x,y)=d(y,x)$.

$$\blacklozenge$$
\newpage

\mysection{46}{Open sets and closed sets}

\renewcommand{\thesubsection}{\thesection.\arabic{subsection}}
\setcounter{subsection}{0}
\subsection{}
\begin{tcolorbox}
In this exercise, let $d$ be the Euclidean metric for $\Breal^2$. Further, let $g$, $m$, $p$, and $h$ be the metrics given in Examples $45.3$ through $45.6$. Let $\theta=(0,\, 0)$. Draw each of the following sets (some will simply be verifications of the pictures given in the text).
\end{tcolorbox}
\textbf{(a)}
\begin{figure}[H]%
    \centering
    \subfloat[$N_d(\theta;1)$ ]{\begin{tikzpicture}
\begin{axis}[ scale=0.8,
xlabel=$x_1$,
ylabel=$x_2$,
axis x line=center, xlabel style={anchor=north west},
axis y line=center, ylabel style={anchor=south west},
xmin=-1.5,
xmax=1.5,
ymin=-1.5,
ymax=1.5,
axis line style={thick, shorten > = -0.5cm, shorten < = -0.5cm},
samples=50,
unit vector ratio*=1 1,
font=\tiny
]
\addplot [domain=-1:1,samples=100, dashed, black, smooth,
pattern={Lines[
                  distance=2mm,
                  angle=45
                 ]},
        pattern color=gray!50]{sqrt(1-x^2)};
\addplot [domain=-1:1, samples=100,dashed, black, smooth,
pattern={Lines[
                  distance=2mm,
                  angle=45
                 ]},
        pattern color=gray!50]{-sqrt(1-x^2)};

%\path[ pattern={Lines[distance=2mm, angle=0]},
       % pattern color=gray!50] (axis cs:0,-7) --(axis cs:0,7)--(axis cs:6,7)--(axis cs:6,-7);
 \end{axis};
\end{tikzpicture}}
    \subfloat[$B_d(\theta;1)$]{\input{./images/fig_3.46.1_a2.tex}}
%\caption{}
\label{fig:fig_p8b}
\end{figure}
\textbf{(b)}
\begin{figure}[H]%
    \centering
    \subfloat[$N_g(\theta;1)$ ]{\input{./images/fig_3.46.1_b1.tex}}
    \subfloat[$B_g(\theta;1)$]{\input{./images/fig_3.46.1_b2.tex}}
%\caption{}
\label{fig:fig_p8b}
\end{figure}
\textbf{(c)}
\begin{figure}[H]%
    \centering
    \subfloat[$N_m(\theta;\half)$ ]{\begin{tikzpicture}[]
\begin{axis}[ scale=0.8,
xlabel=$x_1$,
ylabel=$x_2$,
axis x line=center, xlabel style={anchor=north west},
axis y line=center, ylabel style={anchor=south west},
xmin=-1.5,
xmax=1.5,
ymin=-1.5,
ymax=1.5,
axis line style={thick, shorten > = -0.5cm, shorten < = -0.5cm},
samples=50,
unit vector ratio*=1 1,
font=\tiny
]
%\addplot [domain=-1:1,samples=100,  black, smooth,pattern={Lines[ distance=2mm,angle=45]},
%        pattern color=gray!50]{sqrt(1-x^2)};
\draw[fill = black](axis cs:0,0)circle(3);
\draw [-{Latex[length=1mm]}] plot[smooth , tension=.7] coordinates {(axis cs:0.8,0.8) (axis cs:0.65,0.55) (axis cs:0.3,0.5) (axis cs:0.05,0.05)  };
%\draw[] (axis cs:-1,1)--(axis cs:1,1)--(axis cs:1,-1)--(axis cs:-1,-1)--(axis cs:-1,1){};
%\path[ pattern={Lines[distance=2mm, angle=45]},
 %       pattern color=gray!50] (axis cs:-1,1)--(axis cs:1,1)--(axis cs:1,-1)--(axis cs:-1,-1);
 \node[above] at (axis cs:0.8,0.8)  {$N_m(\theta;\frac{1}{2})=\{\theta\}$};
 \end{axis};
\end{tikzpicture}}
    \subfloat[$N_m(\theta;1)$]{\input{./images/fig_3.46.1_c2.tex}}
        \subfloat[$B_m(\theta;1)$]{\begin{tikzpicture}[]
\begin{axis}[ scale=0.8,
xlabel=$x_1$,
ylabel=$x_2$,
axis x line=center, xlabel style={anchor=north west},
axis y line=center, ylabel style={anchor=south west},
xmin=-1.5,
xmax=1.5,
ymin=-1.5,
ymax=1.5,
axis line style={thick},
samples=50,
unit vector ratio*=1 1,
font=\tiny
]
%\addplot [domain=-1:1,samples=100,  black, smooth,pattern={Lines[ distance=2mm,angle=45]},
%        pattern color=gray!50]{sqrt(1-x^2)};
%\draw[] (axis cs:-1,1)--(axis cs:1,1)--(axis cs:1,-1)--(axis cs:-1,-1)--(axis cs:-1,1){};
\path[ pattern={Lines[distance=2mm, angle=45]},
       pattern color=gray!50] (axis cs:-2,2)--(axis cs:2,2)--(axis cs:2,-2)--(axis cs:-2,-2);
 
 \end{axis};
\end{tikzpicture}}
%\caption{}
\label{fig:fig_p8b}
\end{figure}
\textbf{(d)}
\begin{figure}[H]%
    \centering
    \subfloat[$N_p(\theta;1)$ ]{\input{./images/fig_3.46.1_d1.tex}}
    \subfloat[$B_p(\theta;1)$]{\input{./images/fig_3.46.1_d2.tex}}
%\caption{}
\label{fig:fig_p8b}
\end{figure}
\textbf{(e)}
\begin{figure}[H]%
    \centering
    \subfloat[$N_h(\theta;1)$ ]{\input{./images/fig_3.46.1_e1.tex}}
    \subfloat[$B_h(\theta;1)$]{\input{./images/fig_3.46.1_e2.tex}}
%\caption{}
\label{fig:fig_p8b}
\end{figure}
$$\blacklozenge$$

\newpage
\subsection{}
\begin{tcolorbox}
Using Exercise $1$ or other examples, show that for some metrics
it need not be so that $cl (N(p;\epsilon)) = B(p;\epsilon)$. Further, show that $N(p;\epsilon)$ need not be a proper subset of $B(p;\epsilon)$.
\end{tcolorbox}
Consider the space $\left(\Breal^2,m\right)$ as depicted in $\mathbf{(c)}$ above. $N_m(\theta; 1) =\{\theta\}$ has no limit point. Hence $cl\left(N_m(\theta; 1)\right) =\{\theta\}$, yet $B_m(\theta;1)=\Breal^2$ and thus $N_m(\theta; 1) \neq B_m(\theta; 1) $.\\
$N(p;\epsilon)$ need not be a proper subset of $B(p;\epsilon)$ as can be seen in case $\mathbf{(e)}$. There, $N_h(p;\epsilon)=B_h(p;\epsilon)=\Breal^2$, so $N_h(p;\epsilon)$ is not a proper subset of $B_h(p;\epsilon)$.
$$\blacklozenge$$


\subsection{}
\begin{tcolorbox}
Prove that if $\left(X,d\right)$ is a metric space, $p \in  X, \Et \epsilon > 0$ , then $N(p;\epsilon)$ is an open set. Is it possible in some cases that $N(p;\epsilon)$ is also a closed set
\end{tcolorbox}
We have to prove that
$$\foral q\in N(p;\epsilon),\,\exist \delta>0: N(q;\delta)\subset N(p;\epsilon)$$
Be $q\in N(p;\epsilon)$ and $d(p,q)=\tilde{\epsilon}$. As $q\in N(p;\epsilon)$ we have $\tilde{\epsilon}<\epsilon$.\\
Be $q^*\in N(q;\delta)$ with $d(q,q^*)< \delta$ as a consequence. We have by definition of the metric:
\begin{align*}
d(p,q^*)&\leq d(p,q)+d(q,q^*)\\
&= \tilde{\epsilon}+d(q,q^*)\\
&< \tilde{\epsilon}+\delta
\end{align*}
So choosing $\delta \leq \frac{\epsilon -\tilde{\epsilon}}{2}$ gives 
\begin{align*}
d(p,q^*)&< \tilde{\epsilon}+\frac{\epsilon -\tilde{\epsilon}}{2}\\
&= \frac{\epsilon +\tilde{\epsilon}}{2}\\
&< \frac{\epsilon +{\epsilon}}{2}\quad (\text{ as }\tilde{\epsilon}<\epsilon)\\
&= \epsilon
\end{align*}
So every element $q^*\in N(q;\delta)$ will also be an element of $N(p;\epsilon)$ and we can conclude
$$\foral q\in N(p;\epsilon),\,\exist \delta>0: N(q;\delta)\subset N(p;\epsilon)$$
$$\blacklozenge$$



\subsection{}
\begin{tcolorbox}
Are closed balls $B(p;\epsilon)$  necessarily closed sets?
\end{tcolorbox}
No, consider in Exercise $1$ the case $(c)$ with the closed ball $B_m(\theta; \half)$. We have $B_m(\theta; \half)=\{\theta\}$ which is not a closed set.
$$\blacklozenge$$

\subsection{}
\begin{tcolorbox}
Show that if $\left(X,d\right)$ is a metric space, then $X$ can be expressed as the union of a countable collection of neighborhoods with a common center.
\end{tcolorbox}
Choose any point $p_0\in X$ and define a function $f:\Pee\rightarrow \Breal_+$ such that the set $f(i)$ is a not bounded, strict increasing function  (i.e. $f(i+n)>f(i):\foral i,n\in\Pee$ and  $f$ is a bijection between $\Pee \Et \Breal_+$).\\
Consider the collection of neighbourhoods $\mathscr{N}=\{ N\left(p_0;f(i)\right):i\in\Pee\}$.\\
Obviously, $\mathscr{N}$ is countable and $\bigcup \mathscr{N}= X$. Indeed, be $p\in X$ and $\delta = d(p_0,p)$.\\
As $f:\Pee\rightarrow \Breal_+$  is not bounded, we can find a $i\in\Pee$ such that  $f(i)> \delta$ and thus $p$ will be an element in the set $N\left(p_0;f(i)\right)$ and can conclude $X\subset\bigcup \mathscr{N}$. On the other hand every neighbourhood $N\left(p_0;f(i)\right) \in \mathscr{N}$ contains only elements of $X$ and thus $\bigcup \mathscr{N}\subset X$ and get 
$$X=\bigcup \mathscr{N}$$

$$\blacklozenge$$

\subsection{}
\begin{tcolorbox}
Let$\left(X,d\right)$  be a metric space. Prove that if $A \subset B \subset X$, then $cl(A) \subset cl(B)$.
\end{tcolorbox}
If $p$ is a limit point of $A$, then there are elements of any $\epsilon$-ball around $p$ which are elements of $A$ and thus also elements of $B$ (as $A\subset B$). Thus, $p$ is also a limit point of $A$. So $p$ is both an element of $cl(A)$ and $cl(B)$ and thus  
$$cl(A)\subset cl(B)$$
$$\blacklozenge$$

\subsection{}
\begin{tcolorbox}
In each of the following either prove the statement or give a
counterexample:
\begin{align*}
\begin{array}{l l}
(a)&cl (A\cap B) \subset cl(A)\cap cl (B)\\
(b)&cl (A\cap B) = cl(A)\cap cl (B)\\
(c)&cl (A\cup B) \subset cl(A)\cup cl (B)\\
(d)&cl (A\cup B) = cl(A)\cup cl (B)
\end{array}
\end{align*}
\end{tcolorbox}
\textbf{(a)$cl (A\cap B) \subset cl(A)\cap cl (B)$\\}
Be $p$ a limit point of $A\cap B$. This implies $\foral\epsilon >0: \left(N(p;\epsilon)-\{p\}\right)\cap(A\cap B)\neq \varnothing$. We have $C\cap\left(A\cap B\right)= \left(C\cap A\right)\cap B$ and $C\cap\left(A\cap B\right)= \left(C\cap B\right)\cap A$ and can write
\begin{align*}
&\begin{cases}
\foral\epsilon >0: \left(N(p;\epsilon)-\{p\}\right)\cap(A\cap B)\neq \varnothing\\
\foral\epsilon >0: \left(N(p;\epsilon)-\{p\}\right)\cap(A\cap B)\neq \varnothing
\end{cases}\\
\iff\quad &\begin{cases}
\foral\epsilon >0: \left(\left(N(p;\epsilon)-\{p\}\right)\cap A\right)\cap B\neq \varnothing\\
\foral\epsilon >0: \left(\left(N(p;\epsilon)-\{p\}\right)\cap B\right)\cap A\neq \varnothing
\end{cases}\\
\end{align*}
This implies that any $\epsilon$-ball around $p$ has common elements of both $A$ and $B$ and thus $p$ is also a limit pint of $A$ and $B$ and thus an element of $cl(A)\Et cl(B)$. From this we conclude
$$cl (A\cap B) \subset cl(A)\cap cl (B)$$
$$\lozenge$$
\textbf{(b)$cl (A\cap B) = cl(A)\cap cl (B)$\\}
Considering $(a)$ it remains to prove or disprove
$$cl(A)\cap cl (B)\subset cl (A\cap B)$$
Consider this counterexample in $\Breal$:\\
Be $A=(0,1)$ and $B=(1,2)$ giving $cl(A\cap B)=cl(\varnothing)=\varnothing$. Then, $cl(A)=[0,1]$ and $cl(B)=[1,2]$ with $cl(A)\cap cl(B)=\{1\}$, disproving the proposed expression.
$$\lozenge$$
\textbf{(c)$cl (A\cup B) \subset cl(A)\cup cl (B)$\\}
Be $p$ a limit point of $A\cup B$. This implies $\foral\epsilon >0: \left(N(p;\epsilon)-\{p\}\right)\cap(A\cup B)\neq \varnothing$. We have $C\cap\left(A\cup B\right)= \left(C\cap A\right)\cup \left(C\cap B\right)$ and can write
\begin{align*}
&\foral\epsilon >0: \left(N(p;\epsilon)-\{p\}\right)\cap(A\cup B)\neq \varnothing\\
\iff\quad & \foral\epsilon >0: \left(\left(N(p;\epsilon)-\{p\}\right)\cap A\right)\cup \left(\left(N(p;\epsilon)-\{p\}\right)\cap B \right)\neq\varnothing
\end{align*}
meaning that $p$ must be a limit point of $A$ or $B$ and thus an element of $cl(A)\cup(B)$ and we can conclude 
$$cl (A\cup B) \subset cl(A)\cup cl (B)$$
$$\lozenge$$
\textbf{(d)$cl (A\cup B) = cl(A)\cup cl (B)$}
Considering $(c)$ it remains to prove or disprove
$$cl(A)\cup cl (B)\subset cl (A\cup B)$$
If $q$ is an element of $cl(A)\Ou cl (B)$, but not a limit point of any of them, then $q$ must be an element of $A\Ou B$ and thus an element of $A\cup B$, hence also of $cl(A\cup B)$.\\
If $p$ is a limit point of $cl(A)\Ou cl (B)$, then $p$ must be an element of $cl(A)\Ou cl(B)$. Suppose $p$ is a limit point of $A$, then:  
\begin{align*}
& \foral\epsilon >0: \left(\left(N(p;\epsilon)-\{p\}\right)\cap A\right)\cup \left(\left(N(p;\epsilon)-\{p\}\right)\cap B \right)\neq\varnothing\\
\iff\quad &\foral\epsilon >0: \left(N(p;\epsilon)-\{p\}\right)\cap(A\cup B)\neq \varnothing\quad \scriptstyle \text{as }\left(C\cap A\right)\cup \left(C\cap B\right)=C\cap\left(A\cup B\right)\textstyle
\end{align*} 
thus $p$ is a limit point of $A\cup B$ and thus of $cl(A\cup B)$ and can conclude, together with $(c)$
$$cl (A\cup B) = cl(A)\cup cl (B)$$

$$\blacklozenge$$
\newpage

\mysection{47}{Some basic theorems concerning open and
closed sets}

\renewcommand{\thesubsection}{\thesection.\arabic{subsection}}
\setcounter{subsection}{0}